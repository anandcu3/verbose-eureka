\chapter{Introduction}
\label{chapter:intro}
\emph{Automatic Speech Recognition} (ASR) is the process of converting speech into text. ASR is used in a wide range of applications like digital assistants,  captioning audio conversations, etc. ASR systems convert digital speech audio to text transcriptions using pattern matching techniques. Research in this field has become very dynamic in the previous decade with the usage of deep neural networks for ASR. \emph{Deep neural networks} (DNN) are a faction of machine learning models, which consist of multiple layers of neurons and learn to provide required output. Applying deep neural networks for ASR requires speech audio data which also have been transcribed. This is then fed to the deep neural networks to help it to learn new and unseen audio data, and this process is called model training. State-of-the-art ASR systems that use deep learning require large amounts of training data. 

Increasing the data used for training deep neural network models have previously led to substantial increase in accuracy of these models. Also, considering the decreasing costs of data storage and network costs, scaling up training to higher and higher amounts of training data has become an interesting trend over the recent years. Popular cloud based solutions offered by big enterprises use vast amount of resources available to them to train deep learning models that are able to process most of clear audible speech audio with high levels of accuracy. (TODO : add citation?). These large scale experiments are harder to conduct for smaller companies and for academic institutions, because of the need for high amounts of storage and processing power for the process of training which are not easily available. Hence there is a need for optimising training process to reduce the storage footprint, processing power and resources required to train large scale deep learning models.

Large scale training jobs generally use distributed training, which means that multiple processing units are used with the data present in a network location. This introduces new problems like optimising input output read performance and managing local storage etc. These are the problems with respect to infrastructure and engineering related domains. Apart from these there are also research based problems which include convergence issues when using stochastic gradient descent in a distributed setup of training.  
% Introduction tells the motivation, scope, goal and the outcome of the work. Anyone should be able to understand it. The preferred order of writing your master's thesis is about the same as the outline of the thesis: you first discover your problem and write about that, then you find out what methods you should use and write about that.  Then you do your implementation, and document that, and so on.  However, the abstract and introduction are often easiest to write last.  This is because these really cover the entire thesis, and there is no way you could know what to put in your abstract before you have actually done your implementation and evaluation. This means that you have to rewrite them in the end of your work.

% By the way, rarely anyone write the thesis from the beginning to the end just one time, but the writing is more like process, where every piece of text is written at least twice. Be also prepared to delete your own text. In the first phase, you can hide it into comments that are started with \% but during the writing, the many comments should be visible for your helpers, the advisor(s) and supervisor.

% Read the information from the university master's thesis pages~\cite{ThesisInstructions} before starting the thesis.  You should also go through the thesis grading instructions~\cite{ThesisGrading} together with your advisor and/or supervisor in the beginning of your work. This is my master's thesis, and I am very proud of it.  Of course, when I write my \emph{real} master's thesis, I will not use the singular pronoun \emph{I}, but rather try to avoid referring to myself and speak of the research \emph{we} have conducted---I rarely work alone, after all.  Yet, both \emph{I} and \emph{we} are correct, and it depends on the advisor and the supervisor (of course from you, too), which one they would prefer. Anyway, the tense should be active, and passive sentenses should be avoided (especially, writing sentences where the subject is presented with by preposition), so often you cannot avoid choosing between the pronouns. Life is strange, but there you have it. The introduction in itself is rarely very long; two to five pages often suffice. It usually has two subsections with titles Problem statement and Structure of the Thesis, as follows next.


\section{Problem statement}

% Undergraduate students studying technical subjects do not consider typography very interesting these days, and therefore the typographical quality of many theses is unacceptably low.  We plan to rectify this situation somewhat by providing a decent-quality example thesis outline for students.  We expect that the typographical quality of the master's theses will dramatically increase as the new thesis outline is taken into use.


\section{Structure of the Thesis}
\label{section:structure} 

% You should use transition in your text, meaning that you should help the reader follow the thesis outline. Here, you tell what will be in each chapter of your thesis. Often the thesis does not have as many chapters as is in this template. For example, environment and implementation can be combined as well as chapters of evaluation and discussion.  The rest of this thesis is organized as follows. Chapter~\ref{chapter:background} gives the background, etc.

