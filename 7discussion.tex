\chapter{Discussion}
\label{chapter:discussion}
We discuss the research problems laid out in Section \ref{section:rq} and then in Section \ref{section:future}, we go over the next steps in the research related to both the dataset and also related to the model architectures and experiments.

\section{Revisiting the problem statements}

\begin{enumerate}
  \item \textbf{How to enable large-scale training tasks from an engineering point of view? What are the best  methods  to  store data, load data for such training jobs?}
  
  We use TAR archives to store and access data directly from them using WebDataset. This enables sequential access to training data and makes the data loading efficient and compatible with multiple workers in parallel. 
  
  \item \textbf{What is the effect of training data scale on the quality of the acoustic model used?}
  
  Increasing the scale of training data led to huge boost in \acrshort{wer} and other evaluation metrics. It raises a question of whether there is one point of performance beyond which adding more data does not help with improving the model performance. 
  
    \begin{enumerate}
        \item \textbf{Does increasing the scale of training data affect the convergence time of the acoustic model?}
        
        Increasing the scale of training data actually led to \emph{reduction} in the training time to reach the benchmark \acrshort{wer}. This showed the effect of having diverse and large dataset to obtain better model performance.
    \end{enumerate}
    
    
  \item \textbf{How effective are the multi-\acrshort{gpu}s and multiple processing jobs to accelerate the training time and reach convergence quicker? }
    
    Distributed training methods helped not only to reach benchmark \acrshort{wer} quicker, but also to improve the overall performance, of the model. The effective batch size used in these experiments played a huge role in the quality of the models.
    
    \begin{enumerate}
        \item \textbf{Compare synchronous and asynchronous training methods and analyse which of them are suitable for our system setup?} 
        
        Synchronous data parallel training method proved to be well suited for our experiment setup and worked efficiently to reduce the training time and also provided better \acrshort{wer} in than non-distributed training. In contrast, asynchronous training did not fare too well in our experiments, although this could change with further hyperparameter optimization.
    
    
        \item \textbf{What is the speed-up achieved by using these methods for training an acoustic end to end model?}
        
        For synchronous training, a speed-up of close to 2x was observed when training with 4 GPUs in parallel.
    \end{enumerate}
\end{enumerate}

\section{Future Work}
\label{section:future}
\subsection{Dataset related work}
The dataset used is large and this has enabled all the experiments that we have conducted, but the dataset is not without its drawbacks and future work can be done to address some of these inconsistencies. There are phrases like "Operator Instructions" which are discussed in Section \ref{section:bizspeech} which are wrong transcriptions of what is being said in the audio, and this only came to our attention in the later stages of our experiments. Although some preprocessing steps were taken, the dataset needs more thorough cleaning to remove utterances like the above example from the dataset, which can then lead to better models.

The dataset can also be used for a wide variety of other applications like named entity recognition when properly tagged and in other domains to infer business related results based on the speech in the earnings calls. Also, due to the varied nature of the speakers, it can also be filtered and used to train models which are designed to work with a diverse set of speakers.  

\subsection{Further ASR Experiments}
Changing the model architecture can be tried and experiments involving scaling both model and data can be analysed and the correlation between the two scaling methods can be studied in detail. This could help establish a practical formula which can help researchers decide what minimum amount of data will be required to train models with certain number of parameters. Experiments can also focus on deeper and wider layers in the \acrshort{aed} model. Other architectures like transformers\cite{Vaswani2017AttentionNeed}, conformers\cite{Gulati2020Conformer:Recognition} can be explored for large-scale \acrshort{asr} experiments. This could be easy to extend using the same pipeline, but by replacing the model architecture with various other models.

With more time, we would have liked to tune hyperparameters for each type of training and analyse the best word error rates from the different techniques. Currently, the results indicate that methods that work with large batch sizes have fared well and models seem to struggle when batch size is reduced. It will be interesting to see if this holds even after varying the other hyperparameters like the learning rate, adding a learning rate scheduler, different optimizers etc. In all our experiments, the acoustic models are trained end-to-end with punctuations, capitalizations and other special characters. This is one of the biggest unknown in the experiments, about how the model would vary if it was trained with the normalized text instead. 

\subsection{Evaluation Methods}
For evaluating the models, we choose the best validation word error rate checkpoint as the final model to be used with the test set. However, in many cases we observed a sharp decline between the validation word error rates and the test word error rates even when both the sets are unseen and sampled randomly. This can be avoided by techniques like averaging checkpoints over multiple epochs, which have competitive error rates. This method should be used in the future experiments.